\usepackage{graphicx}
\usepackage{titlesec}
\usepackage{amsmath}
\usepackage[german]{datetime}
\usepackage{float}
\usepackage{caption}
\usepackage{fancyhdr}
\usepackage{lastpage}
\usepackage{kantlipsum}
\usepackage[ngerman]{babel}
\usepackage[T1]{fontenc}
\usepackage[utf8]{inputenc}
\usepackage{chngcntr}
\usepackage{tocloft}
\usepackage{titlepic}
\usepackage{array}
\usepackage{setspace}
\usepackage{url}
\usepackage[colorlinks=false]{hyperref}
\usepackage{tabularx}
\usepackage{multicol}
\usepackage{listings}
\usepackage{minted}
\usemintedstyle{friendly}  % Choose a style (friendly, colorful, monokai, etc.)
\usepackage{tcolorbox}
\usepackage{xcolor}
\usepackage{amssymb}
\usepackage{mathrsfs}
\usepackage{trsym,trfsigns}
\usepackage{tikz}
\usepackage{trsym,trfsigns}
\usepackage{pgfplots}
\pgfplotsset{compat=1.18} % Recommended for newerversions



\usepackage{xcolor}
\usepackage[dvipsnames]{xcolor}

% Definition der Grün-Töne
\definecolor{ChapterColor}{HTML}{296153}
\definecolor{SectionColor}{HTML}{3c8974}
\definecolor{SubsectionColor}{HTML}{50b49b}
\definecolor{VerySubsectionColor}{HTML}{50b49b}
\definecolor{TipBoxColor}{HTML}{cce0da}

% Titelfarbe Scheiblechner 3c8974

% Definition der Blau-Töne
\definecolor{LightBlue}{HTML}{EAF7FF}
\definecolor{DarkBlue}{HTML}{005F9E}


%-----Seitenränder-----%
\usepackage[a4paper, left=1cm, right=1cm, top=2cm, bottom=2cm]{geometry}

%-----Spalteneinstellungen-----%
\setlength{\columnsep}{20pt}
\setlength{\columnseprule}{0.5pt} % Dicke der vertikalen Linie
\setlength{\parskip}{1pt}




%-----Titelformatierungen-----%
\titleformat{\chapter}{\sffamily\Large\bf\sffamily\rlap{\color{ChapterColor}\rule[-0.5ex]{\linewidth}{3ex}\vspace{-3ex}}\sffamily\Large\bf\sffamily\color{white}}{\thechapter}{15pt}{}
\titleformat{\section}{\sffamily\large\bf\sffamily\rlap{\color{SectionColor}\rule[-0.5ex]{\linewidth}{3ex}\vspace{-3ex}}\sffamily\large\bf\sffamily\color{white}}{\thesection}{3pt}{}
\titleformat{\subsection}{\sffamily\rlap{\color{SubsectionColor}\rule[-0.5ex]{\linewidth}{3ex}\vspace{-3ex}}\sffamily\sffamily\color{white}}{\thesubsection}{3pt}{}
\titleformat{\subsubsection}{\sffamily\rlap{\color{VerySubsectionColor}\rule[-0.5ex]{\linewidth}{3ex}\vspace{-3ex}}\sffamily\sffamily\color{black}}{\thesubsection}{3pt}{}

%subtopic Title
\newcommand{\subtopic}[1]{\textbf{\sffamily\textcolor{SectionColor}{#1}}}
%pythontopic
\newcommand{\pytopic}[1]{\textbf{\sffamily\textcolor{blue}{#1}}}



%\titleformat{\subsubsection}{\sffamily}{\thesubsection}{10pt}{}
\titlespacing*{\chapter}{5pt}{5pt}{5pt}
\titlespacing*{\section}{5pt}{5pt}{5pt}
\titlespacing*{\subsection}{5pt}{5pt}{5pt}
\renewcommand{\cfttoctitlefont}{\Huge\bf\sffamily}
\renewcommand{\cftloftitlefont}{\sffamily\section}
\renewcommand{\cftlottitlefont}{\sffamily\section}
\addtocontents{lof}{\protect\vspace{-3\baselineskip}}
\addtocontents{lot}{\protect\vspace{-3\baselineskip}}
\addtocontents{toc}{\protect\vspace{-3\baselineskip}}
\captionsetup{font={sf}}
\renewcommand{\normalsize}{\fontsize{8}{8}\selectfont}
\titleclass{\chapter}{straight}

%Python function hint
\newcommand{\pyname}[1]{\textcolor{blue}{\texttt{#1}}}




%-----Kopf- und Fusszeile-----%
\pagestyle{fancy}
\fancyhf{}
\lhead{\today}
\chead{}
\rhead{STOC - Linus Brehm}
\renewcommand{\headrulewidth}{0.1pt}
\lfoot{HSLU T\&A}
\cfoot{}
\rfoot{\thepage}
\renewcommand{\footrulewidth}{0.1pt}
%Kopf- und Fusszeile erzwingen überall
\patchcmd{\chapter}{\thispagestyle{plain}}{\thispagestyle{fancy}}{}{}

%-----Zähler-----%
\newcounter{simplecount}
\setcounter{simplecount}{0}
\renewcommand{\theequation}{\arabic{simplecount}}
\newcommand{\owncount}{\refstepcounter{simplecount}}

\newcounter{simplecount2}
\setcounter{simplecount2}{0}
\renewcommand{\thetable}{\arabic{simplecount2}}
\newcommand{\tabularcount}{\refstepcounter{simplecount2}}

\newcounter{simplecount3}
\setcounter{simplecount3}{0}
\renewcommand{\thefigure}{\arabic{simplecount3}}
\newcommand{\figurecount}{\refstepcounter{simplecount3}}


\newcounter{simplecount4}
\setcounter{simplecount4}{1}
\renewcommand{\theequation}{\arabic{simplecount4}}
\newcommand{\equationcount}{\refstepcounter{simplecount4}}

%-----Abbildungsbeschriftung-----%
\counterwithout{figure}{chapter}

\renewcommand{\cftfigpresnum}{Abbildung }
\renewcommand{\cfttabpresnum}{Tabelle }

\renewcommand{\cftfigaftersnum}{:}
\renewcommand{\cfttabaftersnum}{:}

\setlength{\cftfignumwidth}{2.5cm}
\setlength{\cfttabnumwidth}{2.5cm}

\setlength{\cftfigindent}{0cm}
\setlength{\cfttabindent}{0cm}

\renewcommand{\figurename}{Abbbildung}
\renewcommand{\tablename}{Tabelle}

\setlength{\parindent}{0pt}

%--------Python Code config--------%
\lstset{
  language=Python,                     % Python syntax highlighting
  basicstyle=\ttfamily\small,          % Monospaced font, small size
  backgroundcolor=\color{gray!8},      % Light gray background
  numbers=left,                        % Line numbers on the left
  numberstyle=\tiny\color{gray},       % Gray line numbers
  numbersep=5pt,                       % Distance of numbers from code
  tabsize=4,                           % Tabs = 4 spaces
  showstringspaces=false,              % Don’t mark spaces in strings
  breaklines=true,                     % Wrap long lines
  keywordstyle=\color{blue}\bfseries,  % Keywords in bold blue
  commentstyle=\color{green!50!black}, % Comments in green
  stringstyle=\color{orange},             % Strings in orange
  %identifierstyle=\color{orange},      % Identifiers (variables, function names) in orange
  morekeywords={factorial,pow,perm,self, as, None, True, False, classmethod, staticmethod}, % Extra Python keywords
}

%---Math----------------------------------------%
\let\oldint\int
\renewcommand{\int}{\oldint\limits}
%
\let\oldsum\sum
\renewcommand{\sum}{\oldsum\limits}

% Tip box
\newtcolorbox{tipbox}[2][]{
	%enhanced,
	boxrule=0pt,
	%frame hidden,
    %borderline west={4pt}{0pt}{TipBoxColor},
	colback=TipBoxColor,
	colbacktitle=TipBoxColor,
	fonttitle=\bfseries\sffamily,
	coltitle=black,
	sharp corners,
	title={#2},
	top=0pt,
	%bottomtitlte=0pt,
	#1
}
    
